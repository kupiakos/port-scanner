% !TeX program = xelatex
\documentclass[12pt,titlepage]{article}
\usepackage[margin=1in]{geometry}
\usepackage{lipsum}
\usepackage[utf8]{inputenc}
\usepackage{hyperref}
\usepackage{fontspec}
\usepackage{fancyhdr}
\usepackage{lastpage}
\usepackage{titling}
\usepackage{xcolor}
\usepackage{textcase}
\usepackage{booktabs}
\usepackage{graphicx}
\usepackage{titlepic}
%\usepackage{datetime}
\setmainfont{Linux Libertine O}
\hypersetup{
    colorlinks,
    citecolor=black,
    filecolor=black,
    linkcolor=black!70!blue,
    urlcolor=black!40!blue
}

\def\thecompany{}
\def\company#1{\gdef\thecompany{#1}}
\def\legalus{\MakeTextUppercase{\thecompany}}
\def\theclient{}
\def\client#1{\gdef\theclient{#1}}
\def\legalthem{\MakeTextUppercase{\theclient}}
\newcommand*{\SignatureAndDate}[1]{%
    \vspace{.4in}%
    \par\noindent\makebox[3in]{\hrulefill} \hspace{.5in}\makebox[2.0in]{\hrulefill}%
    \par\noindent\makebox[3in][l]{#1}      \hspace{.5in}\makebox[2.0in][l]{Date}%
}%
\newenvironment{note}{\itshape}{\par\vspace{1em}}

\newcommand{\vulnnote}[2]{\noindent \textit{#1}: #2\\}
\newcommand{\risk}[1]{\vulnnote{Overall Risk}{\textbf{#1}}}
\newcommand{\descr}[1]{\vulnnote{Description}{#1}}
\newcommand{\impact}[1]{\vulnnote{Impact}{#1}}
\newcommand{\recommend}[1]{\vulnnote{Recommendation}{#1}}
\newcommand{\hosts}[1]{\vulnnote{Hosts Affected}{\texttt{#1}}}

\newenvironment{vulnsection}[1]{#1\begin{minipage}{\dimexpr\textwidth-3cm}\setlength{\leftskip}{3.5em}}{\end{minipage}}

\newenvironment{risk-long}[1]{\begin{vulnsection}{\risk{#1}}}{\end{vulnsection}}
\newenvironment{descr-long}[1]{\begin{vulnsection}{\descr{#1}}}{\end{vulnsection}}
\newenvironment{impact-long}[1]{\begin{vulnsection}{\impact{#1}}}{\end{vulnsection}}
\newenvironment{recommend-long}[1]{\begin{vulnsection}{\recommend{#1}}}{\end{vulnsection}}

\newenvironment{vuln}[1]{\subsection{#1}}{}

\newcommand{\msbulletin}[1]{\href{http://technet.microsoft.com/en-us/security/bulletin/#1}{Microsoft Bulletin \MakeTextUppercase{#1}}}

\company{Kevin Haroldsen}
\client{the Behemoth Network}

% Title Page
\title{\vfill\vfill\textsc{Penetration Test Report}\\\large{for Vulnerability Assessment/Penetration Testing}}
% Title Page
\title{\vfill\vfill\textsc{Penetration Test Report}\\%
\large{for \theclient{}}}
\titlepic{\vfill\hrule\vfill\includegraphics[width=2in]{logo}\vfill}

\author{\thecompany}
\pagestyle{fancy}

\fancyhf{}
\renewcommand{\footrulewidth}{1pt}
\lhead{\today}
\rfoot{\thecompany}
\rhead{\thepage}

\begin{document}
\maketitle

\pagenumbering{roman}
\tableofcontents
\clearpage
\pagenumbering{arabic}
\setcounter{page}{1}

\section{Risk Analysis}

I've determined the overall risk to \theclient{} as being \textbf{High} from this penetration test.
With a targeted attack, a malicious agent could gain full access with domain administrator credentials and retrieve proprietary data.

I also individually rate the discovered vulnerabilities based on their likelihood and impact to determine risk.

\begin{vuln}{RDP Remote Code Execution Vulnerabilities}
\risk{High}
\hosts{192.168.207.42}
\descr{The host is missing a critical security update to Windows Remote Desktop.}
\impact{The Remote Desktop in Windows has a vulnerability which could allow execution of code in the context of the logged-on user, or cause a denial of service attack.}
\recommend{Update Windows to install the hotfix described in \msbulletin{ms12-020}.}
\end{vuln}

\begin{vuln}{SMB Server NTLM Multiple Vulnerabilities}
\risk{High}
\hosts{192.168.207.42}
\descr{The host is missing a critical security update to the Windows SMB service stack.}
\impact{Remote attackers may be able to execute abritrary code, cause a denial or service, or bypass authentication with a brute-force method.}
\recommend{Update Windows to install the hotfix described in \msbulletin{ms10-012}.}
\end{vuln}

\begin{vuln}{PHP \texttt{com\_print\_typeinfo()} Remote Code Execution}
\risk{High}
\hosts{192.168.207.101}
\descr{The host is out of date and installed with PHP.}
\impact{Remote code execution can be performed in the context of the web server.}
\recommend{Update to a new version of PHP and Windows.}
\end{vuln}

\begin{vuln}{SMBv1 Remote Code Execution (Shadow Brokers)}
\risk{High}
\hosts{192.168.207.42,192.168.207.101,192.168.207.121,192.168.208.20}
\descr{The host is prone to a remote code execution vulnerability in the SMBv1 protocol.}
\impact{Arbitrary code can possibly be executed in a system context.}
\recommend{Disable SMBv1, and/or block all versions of SMB with the firewall or otherwise.}
\end{vuln}

\begin{vuln}{Multiple PHP Vulnerabilities}
\risk{High}
\hosts{192.168.207.101}
\descr{The host is installed with an old version of PHP that has reach end-of-life and is vulnerable to many known attacks.}
\impact{Successfully exploiting any of the issues can cause a denial of service, ability to read and write abritrary files, execute arbitrary code, and perform buffer overflow attacks.}
\recommend{Upgrade to PHP version 5.6.10 or later.}
\end{vuln}

\begin{vuln}{WordPress \texttt{wp-admin} Multiple Vulnerabilities}
\risk{High}
\hosts{192.168.207.119}
\descr{The host is running a version of WordPress that is prone to multiple vulnerabilities.}
\impact{Remote Code Execution and gaining access to the administrative page of WordPress are both possible for an attacker.}
\recommend{Upgrade WordPress to at least version 2.8.3.}
\end{vuln}

\begin{vuln}{Poor Password Policies}
\risk{High}
\hosts{192.168.207.101,192.168.207.121,192.168.207.122}
\descr{The hosts used passwords that were easy to guess, and/or shared the passwords with operating system accounts.}
\impact{Access to system administrator accounts became possible with password brute force attempts or simple social engineering.}
\recommend{Enforce stronger password policies, and separate system and service accounts.}
\end{vuln}

\begin{vuln}{SQL Injection Vulnerability can reveal accounts}
\risk{High}
\hosts{192.168.207.101}
\descr{The custom Wheatley Labs page is vulnerable to a SQL Injection attack on its username/password database. In addition, passwords are also stored in plaintext that are shared with system accounts.}
\impact{A remote attacker can retrieve usernames and passwords for local accounts on the system, gaining full access.}
\recommend{Do not store passwords in plaintext, separate system and service accounts, and escape SQL input.}
\end{vuln}

\begin{vuln}{HTTP.sys Remote Code Execution Vulnerability}
\risk{High}
\hosts{192.168.207.121,192.168.208.20}
\descr{The host is missing an important security update to its HTTP parsing mechanisms.}
\impact{Remote attackers can run arbitrary code in the context of the current user and perform actions in the security context of the current user.}
\recommend{Update Windows to install the hotfix described in \msbulletin{ms15-034}.}
\end{vuln}

\begin{vuln}{Damn Vulnerable Web App}
\risk{High}
\hosts{192.168.207.119}
\descr{The system has an extremely vulnerable web application with many, many vulnerabilities.}
\impact{Remote code execution, XSS, SQL Injection, and a large variety of vulnerabilities allow a remote attacker to gain complete control of a device.}
\recommend{Do not use this web application.}
\end{vuln}

\section{Conclusion}
Overall, \theclient has many vulnerabilities which may be exploited.
The risks listed above were deemed the most important to disclose and fix, although others do exist.
I highly suggest following the recommendations provided with each listed vulnerability.

\end{document}      
